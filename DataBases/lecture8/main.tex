\documentclass{article}
\usepackage[utf8]{inputenc}
\usepackage[russian]{babel}
\usepackage{marginnote}

\title{Базы данных, лекция 8}
\author{@mikhirurg}
\date{April 2020}

\begin{document}

\maketitle

\section{Надёжность хранения данных}

Необходимо обеспечить доступность данных, мочь в любой момент времени обратиться к данным и получить их за конечное время отклика.

\subsection{Физическая сохранность данных}

\begin{itemize}
    \item Дисковые массивы с избыточностью (RAID-массив, СХД)
    \item Резарвирование аппаратных компонентов и сетей доступа к данным
    \item Центры обработки данных
    \newlineПарковка автомобилей сотрудников должна быть в 600 метров от ЦОД, их строят в сейсмически нейтральных зонах.
    Бывают мобильные ЦОД'ы.
\end{itemize}

\subsection{Логическая сохранность данных}
\begin{itemize}
    \item Сложно индетифицировать нарушения логич. сохранности. 
    \item Невозможно обеспечить целостность данных в каждый момент времени в процессе выполнения задач по их изменению. Логически согласование состояния БД.
    \parПример: Снятие средств с одного счёта и передача на другой. Между двумя действиями произошёл сбой. Деньги исчезли. Система пришла в логически неверное состояние.
\end{itemize}

\newpage

\subsection{Транзакция}
Транзакция - это последовательность действий с БД, в которой либо все действия были успешны, либо не выполняется ни одно из них.
\newline
\marginnote{Похоже на Git-репозиторий} 
Два результата завершения: (Commit; Rollback)
\newlineСвойство транзакции: 
\begin{itemize}
    \item Атомарность
    \item Согласованность
    \item Изоляция
    \par Если запущено несколько конкурирующих транзакций, результат выполнения каждой из них должен быть скрыт от другой
    \item Долговечночть 
    \par Необходимо зафиксировать выполнение транзакции, чтобы последующие возможные сбои не повлияли на состояние данных.
\end{itemize}
Пример: Перевод денег между абонентами сотовой сети
\newline Списать суммы у одного пользователя двумя разными транзакциями 
\newline(В результате одной из транзакций он ушел в минус)

\subsection{Проблемы конкурирующих транзакция}

\begin{itemize}
    \item Проблема потерянного состояния
    \newline (пример про абонента) каждая из транзакций считала баланс, поняля что нужно снять 100р. и в итоге записала 100 ему на счёт.
    \newline В итоге абонент остался с несуществующими 100 рублями.
    \item Проблема "грязного чтения" 
    \newline Читается кортеж другой транзакции, который должен откатиться. То есть транзакция считала неверные данные.
    \item Проблема неповторяемого чтения
    \newline В первый раз, когда было считывание, сумма была одна. А при записи итоговая сумма оказалась другая. Запись в несогласованное состояние. 
    \item Проблема чтения фантомов 
\end{itemize}

\newpage

\subsection{Блокировки}
Можно разделить на явные и неявные блокировки 
\newline Объект блокировки:
\begin{itemize}
    \item Строка
    \item Столбец
    \item База данных
\end{itemize}
Нужно выбирать правильный объект для блокировки
\newline Бывают монопольные и коллективные блокировки.
\newline Удержание транзакций на время блокировки
\newline Таким образом мы решаем проблемы конкурирующих транзакций. 

\subsection{Уровни изоляции транзакций}
\begin{itemize}
    \item Незавершённое чтение
    \newline если транзакция начала изменять данные, другая не должна менять их
    \item Завершённое чтение
    \newline Разрешаем чтение если транзакция завершилась
    \item Воспроизводимое чтение 
    \newline Если одна транзакция начала читать данные, нужно запретить другим транзакциям менять зависимые данные
    \item Сериализуемость
    \newline Нужно обеспечить то, что появление новых кортежей невозможно, пока не будет снята соответствующая блокировка.
    \newline Блокируются объекты, подпадающие под определённое условие.
\end{itemize}
Таким образом мы обеспечиваем изоляцию транзакций.

\subsection{Журналирование транзакций}
В некоторый журнал записывается план транзакции, транзакция выполняется и после выполнения транзакция стерается из журнала (или её часть). 
\newline Если транзакция не завершилась, по журналу производится Rollback до контрольной точки или до транзакции (зависит от организации самой транзакции).
\newline (Обеспечиваем долговечность транзакции).
\end{document}
