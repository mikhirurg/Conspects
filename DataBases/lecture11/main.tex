\documentclass{article}
\usepackage[utf8]{inputenc}
\usepackage[russian]{babel}
\usepackage{marginnote}
\usepackage{graphicx}
\usepackage{float}
\usepackage{makecell}
\usepackage{marginnote}
\usepackage{adjustbox}
\usepackage[table]{xcolor}

\title{Базы данных, лекция 11}
\author{@mikhirurg}
\date{May 2020}

\begin{document}

\maketitle

\section{Предпосылки возникновения NoSQL БД}

\subsection{Увеличение объемов данных}

За 2013-18 год в хранилищах данных было сохранено больше информации, чем за всю предыдущую историю человечества.
\begin{itemize}
    \item Закон удвоения количества данных за 2 года.
    \item "Закон Мура" - перекличка с удвоением данных.
    \item К концу 2020 года совокупный объём данных превысит 40 зеттабайт. 
    \item 40\% этих данных относятся к сфере развлечений.
    \item Только 23\% этих данных потенциально полезны
    \item Только 3\% доступных данных активно используются 
    \item Если в 2005 году объем автоматически генерируемых данных составлял 11\%, то в 2020 уже 40\%.
    \newline \textit{Например логи, автоматические бекапы и тд.}
\end{itemize}

\subsection{Взаимосвязь данных}

\begin{itemize}
    \item Прямые связи:
    \newline Страницы в Интернет, тэги, словари, онтологии и др.
    \item Косвенные связи:
    \newline Связь цифровыз следов, группирование людей по схожести поведения.
\end{itemize}

\subsection{Слабоструктурированная информация(пример)}

\begin{itemize}
    \item Параметры, общие для всей бытовой электроники.
    \item Параметры, общие для вида техники.
    \item Параметры, общие для конкретного класса этого вида техники.
    \item Параметры, характерные для конкретного бренда или модельного ряда.
\end{itemize}

\subsection{Изменение в архитектуре информационных систем}
\begin{itemize}
    \item Многопользовательская архитектура с центральным узлом хранения и обработки данных
    \newline \textit{Мейнфреймы}
    \item Многозвенные клиент-серверные архитектуры (1990-2000)
    \item Сервис-ориентированные архитектуры (настоящее время)
\end{itemize}

\subsection{Общие характеристики NoSQL БД}

\begin{itemize}
    \item Отказ от использования баз SQL
    \item Неструктурированность схемы данных (schemaless)
    \item Использование агрегатов для представления данных
    \item Слабые ACID свойства
\end{itemize}

\subsubsection{Использование агрегатов для представления данных}

\begin{centering}
\begin{tabular}{|m{15em}|m{15em}|}
        \hline
        Данные в нормализованных отношениях & Данные в агрегатах\\
        \hline
        \cellcolor{green!25}
        При обновлении сохраняется целостность данных
        \newline Оптимальная производительность для широкого спектра запросов
        &
        \cellcolor{red!25}
        Необходимо контролировать целостность данных
        \newline Оптимизация только под определённый вид запросов \\ 
        \hline
        \cellcolor{red!25}
        Неэффективность для распределённых хранилищ
        \newline Низкая скорость чтения многосвязных данных
        \newline Несоответствие объектной модели приложения 
        &
        \cellcolor{green!25}
        Большие возможности по оптимизации цтения данных в распределённых хранилищах
        \newline Возможность существенно повысить скорость чтения некоторых агрегатов данных
        \newline Естественное хранение объектов приложения, поддержка атомарности объектов\\  
        \hline
\end{tabular}
\end{centering}

\subsubsection{Слабые ACID свойства}

\begin{itemize}
    \marginnote{Вспомним свойства транзакций}
    \item ACID: Atomicity, Consistency, Isolation, Durability -- 
    \newline Атомарность, Согласованность, Изолляция, Долговечность
    \item BASE: Basic Avaliability, Soft State, Eventual Consistency --
    \newline Базовая доступность, гибкое состояние, согласованность в конечном счёте.
\end{itemize}


\end{document}
